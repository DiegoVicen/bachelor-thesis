\section{State of the Art}

\subsection{Heuristic Search Libraries}

Heuristic Search is a problem solving method that belongs to Artificial
Intelligence \cite{rusell-2003-aima} used in robotics, pathfinding, computer
gaming among other fields. Due to being more appropriate for the performance
that is in general desired, we can find that the most amount of work on
Heuristic Search is done in imperative, fast languages like C++ or Java.\\

In C++, we can find full search frameworks like \emph{HOG2} \cite{hog2},
\emph{Research code for heuristic search} \cite{cpp-search} or \emph{The
  Heuristic Search Research Framework} \cite{goldenberg-2017-framework}. All
these frameworks offer a full set of algorithms and procedures to override the
default implementations (such as cost, heuristic or expansion functions) in
order to adapt the library's behavior to the problem at hand. Also, these
frameworks offer visual representations options; a feature which is indeed the
main goal in the case of \emph{The Heuristic Search Research Framework}, that
offers a general visualization of algorithm behavior instead of domain specific
ones. On the other hand, in Java we can find similar projects like
\emph{Combinatorial Search for Java} \cite{cs4j} or \emph{AIMA}
\cite{java-aima}.\\

Trying to find similar projects in functional languages is more complicated
than that. However, one can find interesting projects like \emph{AIMA} written
in Common Lisp \cite{lisp-aima} or even a complete search framework written in
OCaml \cite{ocaml-search}. In Haskell, however, projects of this size are
nowhere to be found: All the algorithms are distributed in individual packages
with completely different implementations, as well as the data structures used
to perform the search.


\subsection{Haskell}

\newpage