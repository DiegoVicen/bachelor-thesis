\section{Future Work}

This project is, however, far from being considered over. After the author's
main idea of open sourcing the project there is the desire to make it evolve
into a much bigger, common effort. The main reason behind is that there are
many points that may be interesting to work on.\\

The main one is performance; specially the issue encountered with the garbage
collection. Fixing that problem would make the library twice as fast or more,
making it a priority to work on. The main possible solution is on a lower level
than the library itself, though: developing a mutable, efficient heap structure
in Haskell, which is currently non-existent. This work was considered outside
of the scope of the thesis due to its complexity, but it is an interesting
point to focus future works. Also related with the performance of the library
is the fact of the per-node scheme in the closed list of nodes. Although it
is the main mechanism included by the library to provide the $k$ best
solutions, it can be a great computation demand if used under certain
circumstances. It may be necessary to provide some classical, common
closed-list algorithms for some of the cases studied. However, that is not a
trivial solution, and some effort has to be dedicated to find a proper model
that works.\\

Another point to focus future works is the code re-usability. Right now, there
are some parts of the library that include duplicated code due to the different
monadic and pure parts of the library. In the current setup, these two parts
are almost independent due to their type definitions, although the main
structure of both of them have some common parts. Some careful generalization
to these types may succeed in providing common code for both parts of the
library without renouncing to the current type safety.\\

And, of course, the last point to work on is extending the framework. There are
uncountable algorithms that can be added to it, apart from new pieces for
designing new ones. But maybe the greatest part to be improved is the auxiliary
tools for the framework, some of them that were planned ideas but rejected from
the project due to the time constraint of this thesis. Some of those ideas were
logging and report creation out of benchmarks, test suites for correctness, or
notification systems like email or Telegram to report results after long
testing. All these ideas have to be re-evaluated now and decide which ones are
a priority to include.\\

\newpage

%%% Local Variables:
%%% TeX-master: "tfg"
%%% End: